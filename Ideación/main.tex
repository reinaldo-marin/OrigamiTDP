\documentclass{article}
\usepackage[utf8]{inputenc}
\usepackage[spanish]{babel}
\usepackage{listings}
\usepackage{graphicx}
\graphicspath{ {images/} }
\usepackage{cite}

\begin{document}

\begin{titlepage}
    \begin{center}
        \vspace*{1cm}
            
        \Huge
        \textbf{Proyecto Final}
        
        \textbf{Los primeros pasos}
            
        \vspace{0.5cm}
        \LARGE
        Informática II
            
        \vspace{1.5cm}
            
        \textbf{Reinaldo Marín Nieto}


        \textbf{Jonathan Macías Díaz}
            
        \vfill
            
        \vspace{0.8cm}
            
        \Large
        Despartamento de Ingeniería Electrónica y Telecomunicaciones\\
        Universidad de Antioquia\\
        Marzo de 2021
            
    \end{center}
\end{titlepage}

\tableofcontents
\newpage
\section{Preludio}\label{intro}
Antes de llegar a la fase final donde decidimos que juego íbamos a hacer y en que sistema nos íbamos a enfocar, pasamos por varias ideas, una de estas siendo el famoso Bomberman pero con algunas variaciones, como monstruos, donde Bomberman pueda empujar bombas y lograr misiones con un limite de bombas en su equipamiento, mientras se subía de niveles la cantidad de caja se iban a reducir pero la cantidad de monstruos aumentar, haciendo más complicado el juego. Esta idea siendo descartada porque queríamos quizás algo que cambiara también el ambiente de juego, cosa que no se nos ocurrió.

Pero esta idea nos ayudó a enfocar el género del juego que queríamos desarrollar, con un estilo clásico y donde una nave en el espacio, esquivando patrones de disparos del enemigo, iba abriéndose camino contra los enemigos para poder lograr pasar cada fase del juego. Así decidimos crear Origami: To the Space

\section{Glosario} 
Origami: es un arte que consiste en el plegado de papel sin usar tijeras ni pegamento para obtener figuras de formas variadas, muchas de las cuales podrían considerarse como esculturas de papel.\vspace{0.5cm}

Bullet Hell: juegos de disparos tradicionales que se caracterizan porque el jugador debe esquivar cantidades ingentes de balas que aparecen en pantalla en forma de patrones, que se deben memorizar para poder esquivarlas.\vspace{0.5cm}

Shoot'em up: define un género de videojuegos en los que el jugador controla un personaje u objeto solitario, generalmente una nave espacial, un avión o algún otro vehículo, que dispara contra hordas de enemigos que van apareciendo en pantalla. Es un subgénero de los videojuegos de disparos.

\section{¿Qué es Origami: To the Space?} \label{contenido}
Un juego de naves shoot'em up  basado en juegos como Sky Force, donde tú, siendo una nave de papel, tendrás que avanzar hacia el espacio nivel a nivel, con una temática de estilo origami donde la mayoría de props están relacionados o hechos con papel.



\subsection{Jugabilidad}
En el juego, encarnamos el papel de un avión/nave de papel que, desde una vista en tercera persona de manera isométrica, debe dispararle a docenas de objetivos por nivel, además de esquivar patrones de balas al estilo de un bullet hell. Cada nivel tiene 3 fases, de la cual la final es una batalla contra un jefe único, para un total de 5 jefes y por lo tanto, 5 niveles.
\subsection{Escenario}
El videojuego transcurre en 5 escenarios distintos:

1) Casa Familiar

2) Escuela

3) Carretera

4) Estación Espacial

5) Espacio

\subsection{Trama}
Un avión de papel cobra vida en la casa de su creadora, una niña pequeña. A partir de ahí comienza una aventura de ascenso que termina en el espacio, donde salva al mundo de una invasión extraterrestre diminuta.
\subsection{Desarrollo}
El juego es desarrollado por Jonathan Macías y Reinaldo Marín, como proyecto final para la materia Informática II. 




\section{Lenguaje y Plataforma de desarrollo} \label{contenido}
Los pasos específicos para el desarrollo de la actividad se muestran a continuación:

El videojuego es hecho en el lenguaje de programación C++, en la plataforma de desarrollo QT Creator, con sprites y props hechos en la aplicación de edición Adobe Photoshop.

\section{Referencias} 
Ashcraft, Brian, (2008) Arcade Mania! The Turbo-Charged World of Japan's Game Centers, (Kodansha International)

Robinson, Nick (2005). Enciclopedia de Origami: guía completa y profusamente ilustrada de la papiroflexia. Barcelona: Editorial Acento. ISBN 978-84-95376-62-6.

GamerDic (2014). Diccionario online de términos sobre videojuegos y cultura gamer.



\end{document}
